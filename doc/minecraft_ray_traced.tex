\documentclass[]{scrartcl}

\usepackage{cite}

\title{Minecraft: Ray Traced}
\subtitle{Or: A dynamic acceleration structure for voxel worlds}
\author{Marco Jonkers}

\begin{document}

\maketitle

\begin{abstract}
This project attempts to implement a GPU ray tracer in the video game Minecraft, using CUDA.
The ray tracing itself is done in CUDA.
\end{abstract}

\section{Goal}
Similar to other ray tracing research projects by Intel Corporation.
Also shortly explain the benefits of ray tracing versus regular rasterization.
Minecraft's world is made up of voxels, which is nice for ray tracing.

\section{Setup}
Here I explain how the project is set up and the technologies involved.
Minecraft is coded in Java.
The Minecraft Forge project allows me to do two things:
    Listen to specific events using pre-installed hooks
    Edit Minecraft bytecode before it is loaded
Using Java Native Interface (JNI), I can call into C++ code from Java.
The C++ code contains the CUDA kernel.
Data is passed between OpenGL and CUDA using CUDA's graphics interop layer.

\section{Minecraft Rendering System}
There are four geometry groups:
\begin{description}
    \item[Solid]
    \item[Mipped Cutout]
    \item[Cutout]
    \item[Translucent]
\end{description}
Every geometry group has its own vertex buffer.

\section{Challenges}
The challenges of this project include:
\begin{description}
    \item [Ray tracing performance] Because of gameplay.
    \item [Acceleration structure rebuild speed] Because of gameplay.
\end{description}

\section{Static Geometry}
Test for citing \cite{amanatides1987fast}.
\\
I am also citing \cite{ivson2009gpu}.
\\
I am also citing \cite{reinhard2000dynamic}.

\section{Dynamic Geometry}
This section describes ray tracing dynamic objects, such as NPCs.

\section{Benchmarks}


\section{Future work}
Future work is addressed here.

\bibliography{minecraft_ray_traced}{}
\bibliographystyle{apalike}

\end{document}
